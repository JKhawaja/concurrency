\section{Formal Verification}

Formal verification is entirely dependent on the system requirements and design. This is not (per say) an algorithm that can be glued into existing projects, but much more so a library that follows fundamental mathematical principles that can be utilized in designing a CDS system (a system with CDSs within itself).

We could still use a formal verification section in the paper that briefly discusses using the ideas found in the paper (e.g. in the Fabric package) to verify that a certain CDS system (CDS + Accessing Processes) either maintains a certain defined behavior or avoids a certain defined behavior.

The section will describe a rudimentary approach to: defining approaches to verification.

For example, if we have a CDS system that we design using the `fabric` package (or a similar dependency) ...

\subsection{Built-In Verification Methods}

There are already some built-in verification methods in the fabric package. For dependency graphs we check things like:

\begin{itemize}
	\item CycleDetect(): which ensures that the graph is a DAG
	\item TotalityUnique(): which ensures that all UIs are totality-unique
	\item Covered(): which ensures that every node and edge in the CDS has been addressed
	\item CheckVUIDependents()
\end{itemize}